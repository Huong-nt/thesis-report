\makeatletter
\def\maxwidth#1{\ifdim\Gin@nat@width>#1 #1\else\Gin@nat@width\fi}
\makeatother

\chapter{Introduction}

Automated testing has never been an outdated problem in mobile application developing process. There are over 2,500 manufacturer models and over 100 mobile operating system versions \cite{crittercism}. Daily developed mobile applications need to be tested in wide range of phone designs and platforms. These result in enormous amount of testing scenarios and require an effective industrial testing series. Current software-based testing method has shown some advantages but there are still some issues demanded to be dealt with.

The problem of current testing method is that only software aspect is considered. Tester can run test cases perfectly in software regardless hardware's failure like button or touch screen malfunction. However, each phone operating system requires different corresponding testing framework. These factors conspire to make the cross-platform mobile application testing a very challenging task. \nocite{weinman_thesis}

We propose a new approach that in its design, software and hardware testing are more integrated. Applying image processing technologies, our system can detect content of phone screen and produce actions on them. These actions are performed directly on target phone by Delta Robot. Robotics testing gives us a less invasive way of mobile testing that does not require special software on target phone. In addition, the testing series can be applied on any mobile phone operating systems without doubt.

\section{System overview}

The system overview and design are described below in Figure.\ref{fig:system_overview}.
Our system contains two parts: hardware and software. Hardware part includes Delta robot with its controllers. This part takes responsibility for contacting with the mobile phone, as a human doing. The software framework receives the phone's screen and analyzes data for deciding what action to be executed by the Delta robot. These two parts are connected by a controller module which provides a command interface for the software program to communicate with the hardware layer.

In this thesis, I am responsible for the hardware component of our system. Specifically, I developed and manufactured a Delta Robot and provide control API. The testing framework which is designed and constructed by my partner, \textit{Hieu} Nguyen Duc uses my robot controller module to perform the test.

\begin{figure}[H]
	\centering
	\includegraphics[width=\maxwidth{15cm}, keepaspectratio]{Chapters/Fig/system_overview.png}
	\caption{System overview}
	\label{fig:system_overview}
\end{figure}

\section{Goals of the thesis}
In order to develop a Automated Mobile App Testing System based on the Delta robot, the following objectives are proposed:

\begin{itemize}
\item[--] Software applications
	\begin{itemize}
		\item[+] To detect and recognize mobile phone screen's components (button, selection box, text, keyboard).
		\item[+] To generate test scripts compactly and conveniently.
	\end{itemize}
\item[--] Hardware demands
	\begin{itemize}
		\item[+] Choose a Robotic Platforms.
		\item[+] Design and manufacture a Delta robot.
		\item[+] Create control library(\glspl{api}).
		\item[+] To perform a testing process on the robot with gestures: tap, tap and hold, swipe,... on the phone.
		\item[+] Every action must be executed precisely in the target position.
	\end{itemize}
\end{itemize}

\section{Outline of thesis}

In first two chapters, I will be dealing with technologies and tools supporting Delta Robot which include theoretical backgrounds and implementations in the project.

In the third chapter, I present the design of hardware and software of the robot.

And the next chapter describes practical experiments with the system and our evaluation on the results. Those results can be used as a metric to assess the accuracy and reliability of the system.

In general, the whole thesis aims mainly to build mini delta robot for automated mobile app testing system. \nocite{radim_thesis}