\chapter{Experiments and Evaluations}
\label{ch:experiments}

This chapter presents details of the implementations, experiments and evaluations for the proposed models in previous chapter. Section 1 describes the architecture of training system written in Torch framework which was used to train all the models. Before going onto detailed experiments and evaluations, section 2 provides explanation for the metrics used to evaluate models' performances includeng BLUE, METEOR and CIDEr scores. Subsequently, section 3 presents all the experiments along with evaluation and assessment for each of them.


%% -------------------------------------
%% SECTION 3:
%%   Experimental Environment
%% -------------------------------------

%% -------------------------------------
%% SECTION 4:
%%   Experiments
%% -------------------------------------
\section{Experiments and Evaluations}
\label{sec:chap4_experiment}

This section gives the details of all experiments that I have conducted to explore and evaluate different models as well as the effects of different hyperparameters to the models proposed in previous chapter. 

% \subsection{Experiment 1: Determine the right learning rates}

\subsection{Experiment 1: Test robot accuracy}
\subsubsection{Experiment's purpose}

\subsubsection{Experimental setup}
I pratically found that, for the proposed image captioning model, the following settings for optimization methods work relatively well:
	\begin{itemize}
		\item \textbf{SGD}: learning rate $\eta = 1\mathrm{e}{-2}$, no momentum, no weight decay
		\item \textbf{RMSProp}: learning rate $\eta = 5\mathrm{e}{-4}$; $\alpha = 0.8, \epsilon = 1\mathrm{e}{-8}$
	\end{itemize}

\subsubsection{Experimental results and evaluations}

The train and validation losses as well as language evaluation scores during the first $50,000$ iterations for each configuration are shown. 


[Some pictures here]

\textbf{Conclusion:} With both VGGNet and AlexNet as the CNN module of the model, Adam gives the best results in terms of the model convergence rate. Hence, it will be used as the optimization method in the subsequent experiments.

\subsection{Experiment 2: Set alarm}
