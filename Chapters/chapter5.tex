\chapter{Conclusions and Future Work}

\section{Conclusions}
	\subsection{Thesis contributions}
	During the thesis elaboration, I have learned a lot of robot fabrication techniques, robot control likewise application development and testing.

	By successful going through from easy to complicated tests, the contribution of this thesis includes several aspects listed as below:
	\begin{itemize}
		\item[-] The Delta Robot is successfully developed and can be controlled via API.
		\item[-] The Delta Robot can work accuracy and stability.
		\item[-] The implementation of automated mobile testing framework using robotics technology.
		\item[-] The testing result performed on actual device and application.
	\end{itemize}
	\subsection{Limitations}
	Although the system accomplishes almost test cases successfully, my thesis has some limitation are:
	\begin{itemize}
		\item[-] The processor speed is slow due to the problem of data transmission via USB
		\item[-] The robot's frame structure is very weak owing to be built from plastic.
		\item[-] The design has not really optimal, it can be simpler and more effective.
	\end{itemize}

\section{Future Work}
	
	Although experimental results show that the approach in this thesis is reasonably accurate and is promising, many aspects can have further study to improve the quality and practicality:
	\begin{itemize}
		\item[-] Reconstruction of the robot by more hard materials(e.g.\ 6061 Aluminum or steel) It makes the robot have a solid structure and stable operations.
		\item[-] Standalone, all calculations should be solve on the Arduino to increase the efficiency of processing.
		\item[-] Increase accuracy and processing speed.
	\end{itemize}

	From the achievement in this thesis together with expected improvement, we hope that Robotics technology can replace human work, especially in mobile application testing industry.
