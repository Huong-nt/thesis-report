%% ================= ABSTRACT IN VIETNAMSES ===================== %%
\begin{vnabstract}
\selectlanguage{vietnamese} %% Must be presented here, or you will not be able to use Vietnamese in some pages in your thesis
	\indent Ngày nay, với tốc độ phát triển nhanh chóng của các hãng điện thoại mới với rất nhiều chủng loại điện thoại khác nhau được làm ra, cùng với đó là các phần mềm cho điện thoại di động được phát triển hàng ngày, nhu cầu kiểm thử ứng dụng cho điện thoại hàng loạt ngày một lên cao. Hiện nay phương pháp kiểm thử ứng dụng thường được sử dụng là kiểm thử bằng phần mềm tự động. Tuy nhiên phương pháp trên mới chỉ giải quyết được bài toán phần mềm, việc phần cứng ảnh hưởng tới hành vi ứng dụng chưa được xét đến.
	
	Chúng tôi đề ra một cách tiếp cận mới, ứng dụng công nghệ \textit{robotics} trong kiểm thử ứng dụng di động. Hệ thống của chúng tôi sử dụng rô-bốt để tương tác trực tiếp với điện thoại. Các thao tác được thực hiện với độ chính xác cao nhờ rô-bốt delta. Cùng với đó, nội dung trên màn hình điện thoại được đưa vào máy tính và xử lý, nhận diện với công nghệ nhận diện nội dung hình ảnh. Tất cả các hành động này được kiểm soát dưới hệ thống xử lý kịch bản kiểm thử. Điều này giúp chúng ta giải quyết đồng thời bài toán kiểm thử phần mềm và kiểm thử thiết bị di động. Hơn nữa, phương pháp này không đòi hỏi cài đặt phần mềm chuyên dụng lên thiết bị và có thể áp dụng lên nhiều dòng điện thoại cũng như hệ điều hành di động.
	
	Trong khuôn khổ đồ án, tôi đã tiến hành thiết kế, xây dựng phần cứng và phần mềm điều khiển Delta Robot phục vụ cho hệ thống kiểm thử tự động. Nhằm đánh giá mức độ ổn định và chính xác trong quá trình hoạt động, Robot được chạy theo kịch bản kiểm thử lặp lại nhiều lần. Với kết quả đạt được, chúng tôi hy vọng hệ thống kiểm thử của chúng tôi có thể ứng dụng rộng rãi trong công nghiệp kiểm thử thiết bị và phần mềm.

\selectlanguage{english}
\end{vnabstract}